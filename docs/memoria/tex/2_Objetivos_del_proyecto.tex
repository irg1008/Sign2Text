\capitulo{2}{Objetivos del proyecto}

La principal motivación del proyecto es:

\textbf{
  Mejorar la interacción entre personas que necesitan comunicación por signos y personas que no conocen el lenguaje de signos, intentando aumentar la calidad de vida de las primeras.
}


\section{Objetivos teóricos}

\begin{enumerate}
  \item Crear de un modelo de \sigla{DL}\footnote{\textit{Deep Learning}} capaz de clasificar ASL\footnote{\textit{American Sign Language}} a nivel palabra.
  \item Poner en producción un método para que los usuarios puedan probar el modelo de forma libre y transparente.
  \item Generar una \sigla{API} de código abierto para que otros desarrolladores puedan crear plataformas de cliente, con el objeto de aumentar la audiencia y alcance del proyecto.
  \item Hacer disponible y encapsular el modelo en un contenedor Docker para su libre distribución.
  \item Crear de herramientas que permitan el tratamiento de datos y la transformación entre distintos formatos\footnote{\pe: extracción de fotogramas de un vídeo, concatenación de imágenes}.
  \item Estudiar del comportamiento, estructura e hiperparametrización de redes neuronales convolucionales.
  \item Aprender a fondo el uso de \prog{PyTorch}\cite{PYTORCH}, un \loc{framework} para acelerar la creación y prototipado de redes neuronales.
  \item Exportar del modelo entrenado a un formato estándar que facilite la compatibilidad con el mayor número de librerías en distintos lenguajes de programación.
  \item Mantener de una \loc{codebase} que favorezca la integración continua (\loc{linting}\footnote{Voz ingl. Estudio de limpieza, orden, calidad y redundancia}, \loc{formatting}\footnote{Formatear: Correcta estructura y legibilidad} y \loc{type-checking}\footnote{Comprobación de tipados de variables y funciones}) de código, siguiendo así los estándares en contribuciones \loc{Open Source}.
\end{enumerate}


Estos objetivos representan en general la filosofía y los objetivos teóricos del proyecto. Si entramos más en detalle sobre los aspectos técnicos y las \loc{features} que se esperan obtener al finalizar el proyecto, obtenemos los siguientes puntos:

\section{Caracterísitcas esperadas}

\begin{itemize}
  \item el sistema debe ser capaz de inferir resultados del modelo entrenado a tiempo real.
  \item Se desarrollará un modelo fácilmente escalable y adaptable a distintos dataset de cualquier tamaño.\footnote{El formato de los \loc{datasets} debe ser el mismo. Un modelo entrenado para clasificación de imágenes no podrá clasificar en formato vídeo o audio.}
  \item Se realizará una fase de \loc{data augmentation} sobre los datos iniciales.
  \item Se implementarán distintas redes neuronales para los distintos formatos de datos (imagen y vídeo). Deben ser fácilmente refactorizables y mantenibles.
  \item Cada vez que se estudie el uso de un nuevo formato de dataset, se creará una nueva red, manteniendo la usabilidad de la anterior intacta.
  \item A lo largo de las pruebas y entrenamientos de la red, vamos a probar distintos \loc{schedulers}, \loc{optimizers} y \loc{criterions}\footnote{Funciones de cálculo de pérdida} buscando el que mejor se adecúe al formato de los datos y la estructura de la red.
  \item Se mantendrán unas estadísticas a tiempo real de la fase de \loc{training} y \loc{test} mediante el uso de \prog{Tensorboard}, una herramienta analítica que permite mantener un \loc{log} de imágenes generadas en la ejecución; así como gráficas de costes y \loc{accuracies}.
\end{itemize}
