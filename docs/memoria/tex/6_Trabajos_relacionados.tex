\capitulo{6}{Trabajos relacionados}

\section{Sign2Text - \loc{Letter level translation}}

Este proyecto, con el mismo nombre que el mio, tiene 125 estrellas en Github y se trata de un traductor a tiempo real de lenguaje de signos \sigla{ASL} a texto a nivel de letra. Es capaz de traducir las letras del abecedario usando inferencia por webcam. Usa un modelo preentrenado \sigla{ResNet50} y tiene el modelo alojado en una instancia en AWS.

Podemos acceder al repositorio en \url{https://github.com/BelalC/sign2text.git}

\section{ASL Translator}

Esta aplicación, subida al App Store ya que usa el mismo \loc{dataset} que hemos usado en este proyecto, es una \loc{app} de traducción de texto a signos, es decir, todo lo contrario a la nuestra. Asegura:

\begin{enumerate}
  \item Traducir más de 30000 palabras
  \item Ser capaz de traducir frases complejas
  \item Usar un algoritmo que mezcla los vídeos con los signos de forma que muestran oraciones complejas de forma suavizada y sin cortes.
\end{enumerate}

No compite directamente con la idea de este proyecto pero parece un proyecto interesante y con un hueco en el mercado sin competencia.

Se puede acceder a la página de la \loc{app} en \url{https://apps.apple.com/us/app/asl-translator/id421784745}

\section{\loc{ASL to english} - traducción a nivel palabra}

Este proyecto, que cogió mucha fuerza después de aparecer en algún artículo y hacerse ligeramente viral en LinkedIn, tiene 674 estrellas en Github.

En este proyecto se traducen <<6>> palabras de lenguaje de signo a texto; eso sí, ninguna de las palabras se representa con un gesto en movimiento, todas son estáticas. Usa también una red preentrenada y una webcam como entrada para la inferencia.

Como curiosidad, al comienzo de este proyecto, se mantuvo una conversación con la autora por \loc{e-mail} para ver si se podría hacer una colaboración.

Se puede acceder al repositorio del proyecto en \url{https://github.com/priiyaanjaalii0611/ASL_to_English.git}


\section{How2Sign}

How2Sign \bib{Duarte_CVPR2021} es un \loc{dataset} de \sigla{ASL} no solo compuesto de imágenes y vídeos de gestos, sino que también aporta audios, transcripciones, mapas de profundidad y datos de poses.

Todos los vídeos están grabados usando un \loc{chroma key} para mayor facilidad de cambio de fondo y las poses han sido calculadas en 3D en un estudio dedicado. Por esta razón el tamaño del \loc{dataset} aumenta a 290GB.

Con una licencia \sigla{CC} y sin permiso para uso comercial, el \loc{dataset} se puede acceder en \url{https://how2sign.github.io/}