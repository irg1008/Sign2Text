\capitulo{6}{Trabajos relacionados}

En esta sección vamos a repasar trabajos similares al realizado en este proyecto. Se buscan redes neuronales con similar propósito, así como proyectos cuya idea central similar, usen \loc{datasets} similares, etc.

Las caracterísitcas de este proyecto y las que vamos a nombrar para el resto por comparación son:

\begin{itemize}
  \item \textbf{Nombre}: Sign2Text
  \item \textbf{Tipo (Red/\loc{Dataset}/Otro)}: Red
  \item \textbf{Nivel de traducción}: nivel palabra
  \item \textbf{Detección de movimient en gestos}: sí
  \item \textbf{Lenguaje de entrada}: \sigla{ASL}
  \item \textbf{Lenguaje de salida}: texto (inglés)
  \item \textbf{\loc{Dataset}}: \loc{Dataset} de dominio público extraido de \bib{li2020word}. Compuesto de más de 20000 vídeos con +2000 palabras
  \item \textbf{Librería o \loc{framework}}: \prog{PyTorch}
  \item \textbf{Técnicas adicionales aplicadas}: \loc{data augmentation}
  \item \textbf{Estructura o red neuronal aplicada}: \loc{CNN}, propia
\end{itemize}

\section{Sign2Text - \loc{Letter level translation}}

Este proyecto, con el mismo nombre que este proyecto, tiene 125 estrellas en Github y se trata de un traductor a tiempo real de lenguaje de signos \sigla{ASL} a texto a nivel de letra. Es capaz de traducir las letras del abecedario usando inferencia por webcam. Usa un modelo preentrenado \sigla{ResNet50} y tiene el modelo alojado en una instancia en AWS.

\subsection{Datos sobre el proyecto}

\begin{itemize}
  \item \textbf{Nombre}: Sign2Text
  \item \textbf{Tipo (Red/\loc{Dataset}/Otro)}: Red
  \item \textbf{Nivel de traducción}: nivel letra (letras del abecedario)
  \item \textbf{Detección de movimient en gestos}: no
  \item \textbf{Lenguaje de entrada}: \sigla{ASL}
  \item \textbf{Lenguaje de salida}: texto (inglés)
  \item \textbf{\loc{Dataset}}: \loc{Dataset} con imágenes de signos referenciando las letras del abecedario. Dataset no referenciado
  \item \textbf{Red implementada}: No implementada desde cero. Reentrenamiento sobre \sigla{ResNet50}
  \item \textbf{Librería o \loc{framework}}: \prog{TensorFlow}, \prog{Keras}
  \item \textbf{Técnicas adicionales aplicadas}: Ninguna, \loc{dataset} usado ya preprocesado
  \item \textbf{Estructura o red neuronal aplicada}: \loc{ResNet} (preentrenado)
\end{itemize}

Podemos acceder al repositorio en \url{https://github.com/BelalC/sign2text.git}

\section{ASL Translator}

Esta aplicación, subida al App Store ya que usa el mismo \loc{dataset} que hemos usado en este proyecto, es una \loc{app} de traducción de texto a signos, es decir, todo lo contrario a la nuestra. Asegura:

\begin{enumerate}
  \item Traducir más de 30000 palabras
  \item Ser capaz de traducir frases complejas
  \item Usar un algoritmo que mezcla los vídeos con los signos de forma que muestran oraciones complejas de forma suavizada y sin cortes.
\end{enumerate}

Las caracterísitcas técnicas del proyecto son:

\begin{itemize}
  \item \textbf{Nombre}: ASL Translator
  \item \textbf{Tipo (Red/\loc{Dataset}/Otro)}: Aplicación
  \item \textbf{Nivel de traducción}: nivel oración
  \item \textbf{Detección de movimient en gestos}: no
  \item \textbf{Lenguaje de entrada}: texto (inglés)
  \item \textbf{Lenguaje de salida}: \sigla{ASL} compuesto de varios \loc{clips} fusionados para crear un \loc{video stream} sin cortes.
  \item \textbf{\loc{Dataset}}: No se especifica, pero por las imágenes de la aplicación se puede observar el uso de \loc{dataset}
  \item \textbf{Librería o \loc{framework}}: No se especifica
  \item \textbf{Técnicas adicionales aplicadas}: Desconocido
  \item \textbf{Estructura o red neuronal aplicada}: Desconocido
\end{itemize}

No compite directamente con la idea de este proyecto pero parece un proyecto interesante y con un hueco en el mercado sin competencia.

Se puede acceder a la página de la \loc{app} en \url{https://apps.apple.com/us/app/asl-translator/id421784745}

\section{\loc{ASL to english} - traducción a nivel palabra}

Este proyecto, que cogió mucha fuerza después de aparecer en algún artículo y hacerse ligeramente viral en LinkedIn, tiene 674 estrellas en Github.

En este proyecto se traducen <<6>> palabras de lenguaje de signo a texto; eso sí, ninguna de las palabras se representa con un gesto en movimiento, todas son estáticas. Usa también una red preentrenada y una webcam como entrada para la inferencia.

La ficha técnica de este proyecto es la siguiente:

\begin{itemize}
  \item \textbf{Nombre}: ASL to english
  \item \textbf{Tipo (Red/\loc{Dataset}/Otro)}: Red
  \item \textbf{Nivel de traducción}: nivel palabra
  \item \textbf{Detección de movimient en gestos}: no
  \item \textbf{Lenguaje de entrada}: texto (inglés)
  \item \textbf{Lenguaje de salida}: \sigla{ASL}
  \item \textbf{\loc{Dataset}}: Propio
  \item \textbf{Librería o \loc{framework}}: \prog{Tensorflow}
  \item \textbf{Técnicas adicionales aplicadas}: Desconocido
  \item \textbf{Estructura o red neuronal aplicada}: \loc{MobileNet} (preentrenado)
\end{itemize}

Como curiosidad, al comienzo de este proyecto, se mantuvo una conversación con la autora por \loc{e-mail} para ver si se podría hacer una colaboración.

Se puede acceder al repositorio del proyecto en \url{https://github.com/priiyaanjaalii0611/ASL_to_English.git}


\section{How2Sign}

How2Sign \bib{Duarte_CVPR2021} es un \loc{dataset} de \sigla{ASL} no solo compuesto de imágenes y vídeos de gestos, sino que también aporta audios, transcripciones, mapas de profundidad y datos de poses.

Todos los vídeos están grabados usando un \loc{chroma key} para mayor facilidad de cambio de fondo y las poses han sido calculadas en 3D en un estudio dedicado. Por esta razón el tamaño del \loc{dataset} aumenta a 290GB.

\begin{itemize}
  \item \textbf{Nombre}: How2Sign
  \item \textbf{Tipo (Red/\loc{Dataset}/Otro)}: Dataset
  \item \textbf{Nivel de traducción}: nivel palabra
  \item \textbf{Detección de movimient en gestos}: sí
  \item \textbf{Lenguaje de \loc{dataset}}: \loc{ASL}
  \item \textbf{Técnicas adicionales aplicadas}: Grbación con \loc{chroma key}, estudio de pose en 3D, transcripción de audio.
\end{itemize}

Con una licencia \sigla{CC} y sin permiso para uso comercial, el \loc{dataset} se puede acceder en \url{https://how2sign.github.io/}