\apendice{Especificación de Requisitos}

\section{Introducción}

En este anexo se recoge la especificación de los requisitos que definen las caracterísitcas necesarias de nuestro proyecto.

Estos requisitos se han especificado siguiendo la recomendación del estándar \sigla{IEEE 830-1998} \bib{ieee830} que especifica las siguientes características:

\begin{itemize}
  \item \textbf{Corrección}: Todo requisito especificado es correcto si y solo si refleja una necesidad real y una implementación final en el \loc{software}.

  \item \textbf{Ambigüedad}: Todo requisito debe tener una sola interpretación. Para eliminar cualquier ambigüedad se usaran elementos gráficos o notaciones formales.

  \item \textbf{Completitud}: Una especificación de requisitos es completa sí:
        \begin{itemize}
          \item Incluye todos los requisitos importantes del \loc{software} (funcionalidad, ejecución, diseño, interfaces, etc).
          \item Para todas las posibles entradas y situaciones, existe una resuesta definida correctamente.
          \item Aparecen todas las etiquetas en las figuras y diagramas, así como definidos todos los términos y unidades de medida.
          \item En el caso de que no se cumpla el estándar de algún modo en algún apartado, se debe razonar suficientemente el porqué.
        \end{itemize}

  \item \textbf{Verificalidad}: Un requisito es verificable si existe un proceso no muy costoso por el cual una persona o máquina puede comprobar que el \loc{software} cumple dicho requisito.

  \item \textbf{Consistencia}: La especificación se considera consistente sí y solo sí ningún requisito descrito entra en conflicto con otro. Por ejemplo
        \begin{enumerate}
          \item Dos requisitos describen un mismo objeto con términos distinos.
          \item Se describe una función con distintas implementaciones.
          \item Existe un conflicto lógico entre dos acciones o se llega a un punto en el que dos acciones son válidas en el mismo punto temporal.
        \end{enumerate}

  \item \textbf{Clasificación}: Los requisitos se pueden clasificar según su importancia o estabilidad.
        \begin{itemize}
          \item Importancia:
                \begin{itemize}
                  \item Esenciales
                  \item Condicionales
                  \item Opcionales
                \end{itemize}
          \item Estabilidad: según como afecten los cambios en el requisito.
        \end{itemize}

  \item \textbf{Mofificabilidad}: Una especificación es modificable si permite realizar cambios de manera sencilla manteniendo la consistencia y estilo. Para ello es deseable tener un índice o tabla de contenidos accesible y fácil de entender.

  \item \textbf{Explorabilidad}: Se considera explorable a una especificación de requisitos si el origen de cada requisito es claro tanto hacía atrás (origen del requisito) como hacia delante (componentes que realizan requisito).
\end{itemize}

\section{Objetivos generales}

Lso objetivos principales del proyecto son los siguientes:

\begin{enumerate}
  \item
\end{enumerate}

\subsection{Objetivos teóricos}

\begin{enumerate}
  \item Crear de un modelo de \sigla{DL}\footnote{\textit{Deep Learning}} capaz de clasificar ASL\footnote{\textit{American Sign Language}} a nivel palabra.
  \item Poner en producción un método para que los usuarios puedan probar el modelo de forma libre y transparente.
  \item Generar una \sigla{API} de código abierto para que otros desarrolladores puedan crear plataformas de cliente, con el objeto de aumentar la audiencia y alcance del proyecto.
  \item Hacer disponible y encapsular el modelo en un contenedor Docker para su libre distribución.
  \item Crear de herramientas que permitan el tratamiento de datos y la transformación entre distintos formatos\footnote{\pe: extracción de fotogramas de un vídeo, concatenación de imágenes}.
  \item Estudiar del comportamiento, estructura e hiperparametrización de redes neuronales convolucionales.
  \item Aprender a fondo el uso de \prog{PyTorch}\cite{PYTORCH}, un \loc{framework} para acelerar la creación y prototipado de redes neuronales.
  \item Exportar del modelo entrenado a un formato estándar que facilite la compatibilidad con el mayor número de librerías en distintos lenguajes de programación.
  \item Mantener de una \loc{codebase} que favorezca la integración continua (\loc{linting}\footnote{Voz ingl. Estudio de limpieza, orden, calidad y redundancia}, \loc{formatting}\footnote{Formatear: Correcta estructura y legibilidad} y \loc{type-checking}\footnote{Comprobación de tipados de variables y funciones}) de código, siguiendo así los estándares en contribuciones \loc{Open Source}.
\end{enumerate}


Estos objetivos representan en general la filosofía y los objetivos teóricos del proyecto. Si entramos más en detalle sobre los aspectos técnicos y las \loc{features} que se esperan obtener al finalizar el proyecto, obtenemos los siguientes puntos:

\subsection{Caracterísitcas esperadas}

\begin{itemize}
  \item el sistema debe ser capaz de inferir resultados del modelo entrenado a tiempo real.
  \item Se desarrollará un modelo fácilmente escalable y adaptable a distintos dataset de cualquier tamaño.\footnote{El formato de los \loc{datasets} debe ser el mismo. Un modelo entrenado para clasificación de imágenes no podrá clasificar en formato vídeo o audio.}
  \item Se realizará una fase de \loc{data augmentation} sobre los datos iniciales.
  \item Se implementarán distintas redes neuronales para los distintos formatos de datos (imagen y vídeo). Deben ser fácilmente refactorizables y mantenibles.
  \item Cada vez que se estudie el uso de un nuevo formato de dataset, se creará una nueva red, manteniendo la usabilidad de la anterior intacta.
  \item A lo largo de las pruebas y entrenamientos de la red, vamos a probar distintos \loc{schedulers}, \loc{optimizers} y \loc{criterions}\footnote{Funciones de cálculo de pérdida} buscando el que mejor se adecúe al formato de los datos y la estructura de la red.
  \item Se mantendrán unas estadísticas a tiempo real de la fase de \loc{training} y \loc{test} mediante el uso de \prog{Tensorboard}, una herramienta analítica que permite mantener un \loc{log} de imágenes generadas en la ejecución; así como gráficas de costes y \loc{accuracies}.
\end{itemize}


\section{Catalogo de requisitos}

\section{Especificación de requisitos}


